\documentclass[letterpaper,11pt]{article}

\usepackage{latexsym}
\usepackage[empty]{fullpage}
\usepackage{titlesec}
\usepackage{marvosym}
\usepackage[usenames,dvipsnames]{color}
\usepackage{verbatim}
\usepackage{enumitem}
\usepackage[hidelinks]{hyperref}
\usepackage{fancyhdr}
\usepackage[english]{babel}
\usepackage{tabularx}
% \usepackage[sfdefault]{roboto}
\usepackage{helvet}
% \usepackage{HelveticaNeue}

\pagestyle{fancy}
\fancyhf{} % clear all header and footer fields
\fancyfoot{}
\renewcommand{\headrulewidth}{0pt}
\renewcommand{\footrulewidth}{0pt}
\renewcommand{\familydefault}{\sfdefault}

% Adjust margins
\addtolength{\oddsidemargin}{-0.5in}
\addtolength{\evensidemargin}{-0.5in}
\addtolength{\textwidth}{1in}
\addtolength{\topmargin}{-.5in}
\addtolength{\textheight}{1.0in}

\urlstyle{same}

\raggedbottom
\raggedright
\setlength{\tabcolsep}{0in}

% Sections formatting
\titleformat{\section}{
  \vspace{-4pt}\scshape\raggedright\large
}{}{0em}{}[\color{black}\titlerule \vspace{-5pt}]

% Ensure that generate pdf is machine readable/ATS parsable
\pdfgentounicode=1

%-------------------------
% Custom commands
\newcommand{\resumeItem}[2]{
  \item[]\small{
    \textbf{#1}{: #2 \vspace{-2pt}}
  }
}

% Just in case someone needs a heading that does not need to be in a list
\newcommand{\resumeHeading}[4]{
    \begin{tabular*}{0.99\textwidth}[t]{l@{\extracolsep{\fill}}r}
      \textbf{#1} & #2 \\
      \textit{\small#3} & \textit{\small #4} \\
    \end{tabular*}\vspace{-5pt}
}

\newcommand{\resumeSubheading}[4]{
  \vspace{-0.5pt}\item[]
    \begin{tabular*}{0.999\textwidth}[t]{l@{\extracolsep{\fill}}r}
      \textbf{#1} & #2 \\
      \textit{\small#3} & \textit{\small #4} \\
    \end{tabular*}\vspace{-5pt}
}

\newcommand{\resumeSubSubheading}[2]{
    \begin{tabular*}{0.97\textwidth}{l@{\extracolsep{\fill}}r}
      \textit{\small#1} & \textit{\small #2} \\
    \end{tabular*}\vspace{-5pt}
}

\newcommand{\resumeSubItem}[2]{\resumeItem{#1}{#2}\vspace{-4pt}}

\renewcommand{\labelitemii}{$\circ$}

\newcommand{\resumeSubHeadingListStart}{\begin{itemize}[leftmargin=1pt]}
\newcommand{\resumeSubHeadingListEnd}{\end{itemize}}
\newcommand{\resumeItemListStart}{\begin{itemize}[leftmargin=12pt]}
\newcommand{\resumeItemListEnd}{\end{itemize}\vspace{-5pt}}

%-------------------------------------------
%%%%%%  CV STARTS HERE  %%%%%%%%%%%%%%%%%%%%%%%%%%%%


\begin{document}

%----------HEADING-----------------
\begin{tabular*}{\textwidth}{l@{\extracolsep{\fill}}r}
  \textbf{\href{https://cisaacstern.github.io/}{\Large Charles Stern}} & Email : \href{mailto:charlesisaacstern@gmail.com}{charlesisaacstern@gmail.com}\\
  \href{https://cisaacstern.github.io/}{https://cisaacstern.github.io/} & Mobile : \href{tel:+19144505653}{+1-914-450-5653} \\
\end{tabular*}


%-----------EXPERIENCE-----------------
\section{Experience}
  \resumeSubHeadingListStart
      
% --------Multiple Positions Heading------------
%    \resumeSubSubheading
%     {Software Engineer I}{Oct 2014 - Sep 2016}
%     \resumeItemListStart
%        \resumeItem{Apache Beam}
%          {Apache Beam is a unified model for defining both batch and streaming data-parallel processing pipelines}
%     \resumeItemListEnd
%    \resumeSubHeadingListEnd
%-------------------------------------------

      \resumeSubheading
        {Lamont-Doherty Earth Observatory}{Palisades, NY (Remote)}
        {Data Infrastructure Engineer}{May 2021 - Present}
      \resumeItemListStart
        \resumeItem{Pangeo Forge}
          {Lead software engineer for Pangeo Forge \href{https://pangeo-forge.org/}{(https://pangeo-forge.org/)},
          an open-source, modular Python toolkit focused on distributed parallel ETL
          pipelines for transforming gridded weather, ocean, and climate data (NetCDF, GRIB, etc.)
          into analysis-ready Zarr stores.
          }
          \begin{itemize}
            \item Lead development of the core Apache Beam-based SDK (\texttt{pangeo-forge-recipes}) and deployment tooling.
            \item Develop and run ETL pipelines on GCP Dataflow to populate cloud storage with datasets for research initiatives.
            \item Mentor collaborators in the project through design discussions, debugging sessions, and code reviews.
            \item Communicate our work via publications, conference abstracts, blogs, documentation, and coordination meetings.
          \end{itemize}
        \resumeItem{Apache Beam SDK Development}
          {Along with the core SDK, I develop configuration and deployment tooling.}
          \begin{itemize}
            \item \texttt{pangeo-forge-recipes}: Lead development of Apache Beam data transform library, including design/roadmapping, feature development, unit/integration testing, code review, documentation, packaging, and releases.
            \item \texttt{pangeo-forge-runner} CLI: Apache Beam is flexible and powerful, but it's somewhat hard to deploy. This CLI, on which I am core developer, provides a unified interface for execution and storage configuration.
            \item Github Action integration: Developed and released a custom GitHub Action that allows users to deploy ETL pipelines to both test and production environments as part of CI/CD workflows.
            \item Upstream contributions: I currently lead development of the \texttt{DaskRunner} backend for Apache Beam, which encompasses multiple in-flight PRs on both Beam and Dask.
            I also participate in community meetings related to Zarr V3 implementation in \texttt{zarr-python}, and have
            prior PRs merged to \texttt{xarray}, Dask \texttt{distributed}, and \texttt{fsspec}.
          \end{itemize}
        \resumeItem{ETL Pipeline Development \& Deployment}
          {In addition to developing tooling, I also employ that tooling to build large
          datasets on cloud storage (GCS, S3) for research initiatives at Columbia University and for our collaborators.}
          \begin{itemize}
            \item Automated CMIP6 Ingestion Pipelines: Collabrated with Columbia's LEAP-STC initiative to
            customize Pangeo Forge such that it is able to automatically generate pipelines for CMIP6 models,
            resulting in automated builds of 1000s of datasets which pull NetCDF files from ESGF and output Zarr to GCS.
            \item ClimSim: Co-author on this NeurIPS award-winning ML benchmark dataset
            \href{https://arxiv.org/abs/2306.08754}{(https://arxiv.org/abs/2306.08754)}.
            My contributions included designing and deploying a Pangeo Forge pipeline to extract 25 TB of high-res data
            from Hugging Face, and build it into a 25 TB Zarr store on GCS in a single GCP Dataflow batch job.
            \item SWOT Inter-comparison: Built a 2TB catalog of Zarr ocean model data in consisting of
            regional subsets of the GIGATL, HYCOM25, HYCOM50, eNATL60, FESOM, ORCA36, and FIO-COM32 models,
            which formed the data basis for the following publication: \href{https://doi.org/10.5194/gmd-2022-27}{https://doi.org/10.5194/gmd-2022-27}.
            The biggest challenge here was not data size but rather maintaining connections with the flaky academic FTP servers in Europe
            that hosted the source data!
            \item NASA PACE: Collaboration with scientists at UW's Applied Physics Lab.
            Built a $\sim$800 GB Zarr store onto the NSF Open Storage Network from a subset of fields provided by the Aqua MODIS mission
            as part of work to benchmark ML models being developed for analysis of forthcoming data from
            NASA's PACE ocean color mission.
            \item In addition to above large projects, I have also lead and/or collaborated on pipelines for a range of other datasets
            including CESM POP, NOAA HRRR, NOAA OISST, NASA SMAP, NASA GPCP, and NASA GPM IMERG.
          \end{itemize}
        \resumeItem{REST API Service Development \& Operation}
          {Developed and operated a backend service for Pangeo Forge Cloud,
          a managed ETL service, built with FastAPI and deployed to Heroku,
          with additional components deployed to via Terraform to GCP.
          Docs archived here: \href{https://github.com/pangeo-forge/pangeo-forge-orchestrator/blob/main/docs/README.md}
          {https://github.com/pangeo-forge/pangeo-forge-orchestrator/blob/main/docs/README.md}
          }
          \begin{itemize}
            \item FastAPI application with two roles: CRUD interface for Postgres
            database storing application state and data catalog; and orchestration point for translating GitHub App
            event payloads into deployed ETL jobs on GCP Dataflow. 
            \item Docker: Customized a production Docker image using a multi-stage build to optimize for efficient deployment size.
            \item Infrastructure as Code (IaC): Developed custom solution for automated GCP Dataflow job monitoring.
            Deployed with Terraform, this solution included GCP Cloud Functions, Logging Sinks, and Pub/Sub topics,
            and sent webhooks back to our core FastAPI application with notifications when jobs either succeeded or failed.
            \item CI/CD: Deployment pipeline included review, staging, and production applications, each provisioned with its
            own Postgres database and Terraform-deployed job monitoring instance. This allowed builds of the full stack to
            be integration-tested at every phase of deployment, catalyzing rapid iterative development.
          \end{itemize}

      \resumeItemListEnd

    \resumeSubheading
      {Career transition: self-directed study}{Santa Monica, CA}
      {Project-based learning in geospatial data engineering}{Summer 2012 and 2013}
      \resumeItemListStart
        \resumeItem{Python fundamentals}
          {Created models for portfolio hedging,  portfolio optimization and price forecasting. Also creating a strategy backtesting engine used for simulating and backtesting strategies.}
        \resumeItem{UCSB/CREEL albedo research project}
          {}
        \resumeItem{Coursework}
          {}
      \resumeItemListEnd
    \resumeSubheading
      {FormaPath (formerly Parker Isaac Instruments)}{Ithaca, NY}
      {Co-Founder and Chief Product Officer}{Jan 2015 - Aug 2020}
      \resumeItemListStart
        \resumeItem{Product-market fit}
          {}
        \resumeItem{Clinical evaluation, regulation}
          {}
        \resumeItem{Electromechanical medical device development}
          {}
      \resumeItemListEnd

  \resumeSubHeadingListEnd

%-----------EDUCATION-----------------
\section{Education}
\resumeSubHeadingListStart
  \resumeSubheading
    {Bowdoin College}{Brunswick, ME}
    {Bachelor of Arts in Religion and East Asian Studies}{Sept. 2005 -- May 2009}
\resumeSubHeadingListEnd

%-----------PROJECTS-----------------
\section{Programming Skills}
  \resumeSubHeadingListStart
    \resumeSubItem{Languages}
      {Python - Advanced(proficient?); JavaScript (React, NEXT) - Basic contribution proficiency; HTML/CSS; LaTeX}
    \resumeSubItem{Testing/CI}
      {Pytest, Github Actions}
    \resumeSubItem{Deployment}
      {Terraform, Docker, GCP Cloud Functions, GCP Cloud Run, GCP Dataflow}
    \resumeSubItem{Storage}
      {GCS, AWS S3}
    \resumeSubItem{Documentation}
      {Docstrings, Sphinx, MyST Markdown; e.g. \href{https://pangeo-forge.readthedocs.io/}{https://pangeo-forge.readthedocs.io/}}
  \resumeSubHeadingListEnd


%-------------------------------------------
\end{document}
